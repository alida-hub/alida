\subsection{Configuring \alida}
\label{subsec:configure-user}

Sometimes it is desirable to configure 
some properties of \alida or the general behavior of specific operators at
runtime, e.g., to specify initial files or directories where operators should work on. 
\alida basically supports two different ways for user specific configuration:
\begin{itemize}
    \item[a)] environment variables
    \item[b)] properties of the Java virtual machine specified with the \\
    	option \texttt{'-Dproperty=value'} upon invocation of the JVM
\end{itemize}
This order already reflects the priority of the options, i.e., environment
variables overwrite JVM properties.
If for a certain requested property no configuration values are provided by any of these ways,
default settings are used.
Some variables of general interest are used by \alida and are summarized
below.
Further variables may be introduced, e.g., by additional operators implemented in the framework.

In the following list, both the environment variable and the name of the
property are given in the form of \icode{property / environment variable}:
\begin{description}
 \item[\icode{alida.oprunner.level / ALIDA\_OPRUNNER\_LEVEL}] ~

	Used by the graphical operator runner \icode{ALDOpRunnerGUI} and Grappa,
	i.e., the application \icode{ALDGrappaRunner}, to configure which set of
	operators is to be displayed initially in the selection menu.
	Possible options are either all available operators ('standard')             
	or just the ones categorized as being easier to use ('application').
	The default is 'application'.

 \item[\icode{alida.oprunner.favoriteops / ALIDA\_OPRUNNER\_FAVORITEOPS}] ~
	Holds a colon separated list of filenames. Each file
	contains lines of fully qualified operator names which will be unfolded
	in the operator selection window when starting the graphical user interface
	or \grappa (see Sec.~\ref{subsec:userGUI} and~\ref{subsec:grappa}).
	The default is \icode{\$\{user.home\}/.alida/favoriteops}.

 \item[\icode{alida.oprunner.operatorpath / ALIDA\_OPRUNNER\_OPERATORPATH}] ~
	Here a colon separated list of packages may be specified.
	Each package and all its sub-packages 
	are searched for operators in the classpath.
	These operators are incorporated in the tree of available operators
	in the graphical user interface
    and in \grappa.
	This feature is useful to incorporate operators which are not compiled,
	but just added within a jar-archive.
	
 \item[\icode{alida.oprunner.workflowpath / ALIDA\_OPRUNNER\_WORKFLOWPATH}] ~

	Here a colon separated list of directories may be specified each of which
	is searched for workflows saved in a file.
	These workflows are incorporated in the tree of available operators
	in the graphical user interface
        in \grappa.
	The default is \icode{\$\{user.home\}/.alida/workflows}.
	

	
 \item[\icode{alida.versionprovider\_class / ALIDA\_VERSIONPROVIDER\_CLASS}] ~

 Implementation of \icode{de.unihalle.informatik.Alida.ALDVersionProvider} to be used
 for version information retrieval in process documentation
 (Sec.~\ref{subsec:version}).
 
 \item[\icode{alida.version} / --- ] ~
 
 Only available as JVM property, this variable is used by
 \icode{ALDVersionProviderCmdLine} to discover the software version to be stored
 in the history. The class \icode{ALDVersionProviderCmdLine} implements the
 \icode{ALDVersionProvider} interface and gets the version to be used from the
 JVM.
\end{description}

