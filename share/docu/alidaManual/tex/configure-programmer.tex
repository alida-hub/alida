\subsection{Configuring \alida}
\label{subsec:configure-programmer}

As described in Sec~\ref{subsec:configure-user}, \alida allows to configure
some of its properties and general behaviours at run-time. To grant
programmers easy access to this functionality for their own operators as
well, the \alida library includes helper classes with a flexible API for
run-time configuration from the environment. In particular, it provides the class
\icode{ALDEnvironmentConfig} in the package \icode{de.unihalle.informatik.Alida.helpers}
to ease run-time configuration via the probably most common ways of individual
configuration, i.e., in terms of environment variables and properties of the
Java Virtual Machine.

The most general method, available in two versions, in class
\icode{ALDEnvironmentConfig} is 
\vspace*{0.5cm}
\begin{code}
	public static String getConfigValue( \
												String prefix, String operator, String propname)

	public static String getConfigValue( String operator, String propname)																
\end{code}

\vspace*{-0.25cm}
These methods retrieve corresponding values from either environment variables or
JVM properties, in exactly this order. The strings passed as arguments to the
routines are concatenated and properly formatted according to the usual naming
conventions in \alida and the operating system, respectively
(cf.~Sec.~\ref{subsec:configure-user}). In detail, given a {\em prefix},
{\em operator} and {\em property}, the method is searching for an environment
variable named \icode{'PREFIX\_OPERATOR\_PROPERTY'}, or subsequently for a JVM property named
\icode{'prefix.operator.property'}. Note that in the \alida context the string
\icode{'prefix'} in the first method signature should usually be set to
\icode{'alida'}. In the second method, where the prefix argument is missing,
this is done by default.

If the way how a certain property is specified, i.e., either as JVM property or
environment variable, is known in advance, specific methods can be used:
\vspace*{0.5cm}
\begin{code}
	public static String getEnvVarValue( String operator, String propname) 
					
	public static String getJVMPropValue(
													String prefix, String operator, String propname)
																				
	public static String getJVMPropValue(String operator, String propname)	
\end{code}

\vspace*{-0.25cm}
The first one just looks for environment variables, always assuming the prefix
\icode{'ALIDA'}. The other two methods are exclusively accessing JVM properties
where the first call allows to specify a custom prefix, while the second one
assumes \icode{'alida'} by default.
																										
Note that if for a certain requested property no configuration values are
provided by any of these ways, all methods return \icode{'null'} values and the
requesting classes are supposed to fall back to default settings as
defined by the programmer of an operator. In general, there is no limitation for an operator to access
configuration variables of its choice. Usually they should just be properly
documented in the Javadoc of the corresponding operator class.

The naming of environment variables and properties is not strictly regularized
and left free to the programmer of an operator. However, it is strongly
recommended to adhere to the \alida naming conventions as this helps to avoid
name clashes. In particular, have your variables start with prefix
\icode{'alida'} and let the second part be the name of the class or operator
using the variable. The third part is then the actual property. Make sure that
new variables and properties do not collide with variables predefined in \alida
which are listed in Sec~\ref{subsec:configure-user}.


