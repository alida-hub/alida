\subsection{Data I/O provider}
For the generic execution of operators via command line or graphical user
interfaces in \alida, knowledge is required about how to perform input and
output operations for parameter data objects.
In particular, in graphical contexts \alida requires functionality to query 
values for parameter objects from the user via graphical input components, and
to adequately visualize parameter values graphically. For invocation of operators from 
command line, parameter data objects need to be instantiated from textual input,
and parameter objects need to be transformed into a user-friendly textural representation. 

To enable flexible data I/O \alida implements an easily extendable provider mechanism. 
A {\em provider} is a class implementing the functionality to perform I/O for
objects of specific data types, i.e., to instantiate objects from given user
input and visualize their values graphically or in a textural fashion. These
providers are managed by {\em data I/O managers} which keep track of available
providers and provide methods for getting provider objects for specific data
types at run-time, e.g., to facilitate the automatic generation of graphical user interfaces.

Currently, two user interface contexts are implemented in \alida, i.e., a tool
for executing operators from command line (cf.~Sec.~\ref{subsec:userCmdline})
and a graphical operator runner based on Java Swing (cf.~Sec.~\ref{subsec:userGUI}). For both contexts \alida already offers 
built-in providers for a large variety of different parameter data types. In detail, all primitive 
data types available in Java (e.g., int, double, boolean) as well as corresponding wrapper data 
types (e.g., Integer, Double, Boolean), enumerations, arrays of primitive and wrapper data types, 
and also all kinds of Java collections\footnote{The only exception are
collections of collections which are not yet supported.} are supported
out-of-the-box.
Furthermore, operators may be used as input parameters of other operators, and by the concept 
of parameterized classes (Sec.~\ref{subsubsec:datatypes}) \alida allows for easy extensibility. In some cases,
however, it might be necessary to implement a data I/O provider for a certain 
parameter data type from scratch. 
Below we outline the basics of how to accomplish this.
 
All providers in \alida are registered by I/O managers upon start-up. These managers basically
keep a map of data types and related providers and allow the framework to get a provider object
for a certain parameter data type at run-time. 
Additionally, they may give hints to the provider classes to adapt their
functionality to the current mode of operation.
The type of these hints is specific to the context, e.g., command line or GUI,
and will be described in the following.

To enable the managers to automatically register 
all provider classes available on the classpath, it is necessary to annotate the
provider classes with the \alida annotation \icode{@ALDDataIOProvider}. 
This annotation features the field \icode{'priority'} which is used to
select a provider in case that several providers handle the same data type.
This allows, e.g., to override providers delivered by \alida with custom made providers.
In addition, all providers have to 
implement the interface 
\vspace*{0.5cm}
\begin{code}
	de.unihalle.informatik.Alida.dataio.provider.ALDDataIO 
\end{code}

\vspace*{-0.25cm}
It basically defines a single method that all providers need to implement:
\vspace*{0.5cm}
\begin{code}
	public Collection<Class<?>> providedClasses();
\end{code}

\vspace*{-0.25cm}
This method is used by the I/O managers upon start-up to gather information about
which classes a specific provider supports, i.e., to fill its maps. Of course,
apart from this method the actual functionality of a provider class depends on the user interface 
context in which it is to be used. In the following subsections we will now describe the specifics
of implementing providers for the Java Swing (Sec.~\ref{subsubsec:implProviderSwing}) and 
the command line context (Sec.~\ref{subsubsec:implProviderCmdline}).
  
\subsubsection{Implementing a Swing data I/O provider}
\label{subsubsec:implProviderSwing}
To enforce all providers dedicated to the Swing context to implement the corresponding graphical 
data I/O concept, \alida defines a specific interface for the Swing context: 
\vspace*{0.5cm}
\begin{code}
	de.unihalle.informatik.Alida.dataio.provider.ALDDataIOSwing 
\end{code}

\vspace*{-0.25cm}
This interface subsumes all methods to be implemented by Swing providers.

\paragraph{Graphical data input.} The basic workflow which
\alida defines for graphical input of a parameter data object is as follows. First a graphical 
component needs to be generated by the provider suitable for querying values 
from the user. Such a component is for example given by a checkbox to query the user for a boolean 
value, a textfield to query for a string or number, or a combobox to let the user select a single
or multiple values from a larger collection. This component is then included in graphical user
interfaces, e.g., the graphical control windows of \alida's graphical operator runner.
Subsequently, once the user has entered appropriate values, functionality is required 
to read the specified values from the graphical component. 

Accordingly, the interface basically defines the following two methods:
\vspace*{0.5cm}
\begin{code}
  public abstract ALDSwingComponent createGUIElement(
      Field field, Class<?> cl, Object obj, ALDParameterDescriptor descr);

  public abstract Object readData(
      Field field, Class<?> cl, ALDSwingComponent guiElement);
\end{code}

\vspace*{-0.25cm}
The first method is supposed to return a graphical component suitable for being included in 
graphical control windows. In general, \alida does not define any strict rules
for these components except that they have to be of type
\icode{ALDSwingComponent} and implement the methods defined in that interface.
Objects of this type basically wrap an object of type \icode{JComponent}. 
Only the latter one is actually included in the configuration and control windows. 
Although there are principally no restrictions on the 
design of the components returned by providers, note that proper automatic GUI
layout is only possible if the different 
components do not vary too much in their sizes. Hence, it is advisable to layout the components 
in such a way that they only fill one row in a panel. Elements which naturally obey to this rule 
are for example buttons, text fields, and combo- or checkboxes. If you require more complex 
components to query values from the user, one solution could be to return a button as main GUI
component by which a new window can be opened with an arbitrary size where the actual parameter 
values can then be entered. 
This technique is employed, e.g., by \alida's standard providers for
parameterized classes and collections.

Besides wrapping a graphical component, objects of type
\icode{ALDSwingComponent} have to implement an event reporter interface. This
interface lays the foundation for \alida to be able to visualize the current state of an operator's configuration in its control windows in real-time. The interface
basically enforces the graphical component to trigger events on changes of the values specified
in the graphical component. For most providers relying on Swing components as graphical components
it is sufficient to implement listeners for the Swing components and simply forward Swing events
as \alida events of type \icode{ALDSwingValueChangeEvent} to the framework.
Refer to the API documentation of
\icode{de.unihalle.informatik.Alida.dataio.provider.swing.components.ALDSwingComponent}
and related event reporter and handler classes in the package
\icode{dataio.provider.swing.events} for more details. Also note that
\alida already provides implementations of graphical elements for data
I/O of common data which subclass \icode{ALDSwingComponent}. They are to be
found in the package \icode{dataio.provider.swing.components} and can be used
out-of-the-box.

The method \icode{createGUIElement(field, cl, obj, descr)} takes four
arguments as input.
The argument \icode{'cl'} specifies the class of the parameter object which 
should be read by the component (which is of special interest for providers supporting various
data types), while the \icode{'obj'} argument allows to pass a default value to
the provider. The argument \icode{'descr'} allows to hand over additional
information to the provider about the operator parameter with which the object is associated. \alida uses this information, e.g., for 
generating more meaningful labels in the graphical control windows. Note that if no descriptor is
available, \icode{'null'} is also a valid value, hence, providers must account
for this specical case.
The first argument \icode{'field'} can be ignored by most
providers. There are only a few providers where the object class is not sufficient 
to instantiate a parameter object, and where in addition the field associated
with the member variable representing the parameter is required to have all relevant information available. 
One example are collections in Java, where the data type of the
elements contained in the collection is not obvious from the class of the collection itself.

The second method \icode{readData(\ldots)} of the \icode{ALDDataIOSwing} interface takes as argument a 
formerly generated graphical
component \icode{'guiElement'} and is supposed to return an object of the
specified class. The object should
contain the value as currently specified by the graphical component.
The \icode{'field'} argument can most of the time be ignored by the provider,
refer to the previous paragraph for details. 

In addition to the formerly discussed methods the interface defines a third method
\vspace*{0.5cm}
\begin{code}
  public abstract void setValue(
      Field field, Class<?> cl, ALDSwingComponent guiElement, Object value);
\end{code}

\vspace*{-0.25cm}
which is supposed to set new values in the graphical component. Its arguments are quite similar 
to the \icode{createGUIElement(\ldots)} method. The new value to be set is
specified by the \icode{'value'} argument.

\paragraph{Graphical data output.} For performing graphical data output the
Swing provider interface defines the method
\vspace*{0.5cm}
\begin{code}
  public abstract JComponent writeData(Object obj, ALDParameterDescriptor d);
\end{code}

\vspace*{-0.25cm}
It basically takes an object \icode{'obj'} as input and generates a suitable
Java Swing component to properly visualize the object's value. Note that the class of the object is directly derived from 
the object itself. The additional descriptor argument is used to enhance the graphical component
with more information about the object. The object might be \icode{'null'},
i.e., the provider has to check its value prior to accessing the object
directly. For components generated by this method the same rules hold as for the graphical input components. In particular, ensure that each component does not exceed the height of one row in a panel to enable proper GUI layout.

\paragraph{Interaction level.} 
The I/O manager for the Swing context gives providers a hint on the amount of
user interaction level the providers should generate.
The reason is that 
there are situations when only warnings should be signaled to the user or no
pop up of windows and user interaction is intended at all. To request the
desired level of interaction and to modify the setting, the Swing data I/O
manager offers the following methods:
\vspace*{0.5cm}
\begin{code}
 public ProviderInteractionLevel getProviderInteractionLevel()

 public void setProviderInteractionLevel(ProviderInteractionLevel level)
\end{code}

\subsubsection{Command line provider}
\label{subsubsec:implProviderCmdline}

In analogy to Swing providers each command line provider is required to
implement the interface
\vspace*{0.5cm}
\begin{code}
	de.unihalle.informatik.Alida.dataio.provider.ALDDataIOCmdline
\end{code}

\vspace*{-0.25cm}	
This interface defines two methods to be implemented, namely
\vspace*{0.5cm}
\begin{code}
  public abstract Object readData(Field field, Class<?> cl, String valueString);

  public abstract String writeData(Object obj, String locationString);
\end{code}

\vspace*{-0.25cm}
The first method is expected to instantiate an object of class \icode{'cl'} from
the string \icode{'valueString'}.
For the meaning of the argument \icode{'field'} see the subsection
on Swing providers.
In general, the syntax of the value string depends on the data type
and may freely be defined  by the implementation of the provider.
However, \alida offers the notion of a standardized command line provider
taking care of derived classes as  well as reading and writing to and from files
(see below).
Additionally, \alida provides a command line provider for parameterized classes
featuring a specific syntax (see Sec.~\ref{subsubsec:datatypes}).  
If appropriate, it may be sensible to adopt the syntax for other
providers as well.

The method \icode{writeData(\ldots)} is used to format the given object to textual
form into a string.
The \icode{'locationString'} specifies whether this textual representation is
to be returned as the result of the method or should be written to, e.g., a
file.
In the latter case the string returned may contain information of this process,
e.g., return a note that the object was written to a specific file.
As for the syntax of  the \icode{'valueString'} in \icode{readData(\ldots)}, the
syntax of the  \icode{'locationString'}  may in principle be freely defined by each
provider.
However, it is advisable to adhere to \alida's standard definitions by the
standardized command line provider as described in
Sec.~\ref{subsec:userCmdline}.
In addition, the \icode{'locationString'} may also be used by a provider as
a format string to modify the textual representation of the object generated.


\alida features a so called {\em standardized commandline provider}
to generically handle data I/O from and to file, and for the input of derived
classes, in the abstract class \icode{ALDStandardizedDataIOCmdline}.
It is easy to implement new providers extending this class and,
thus, to automatically inherit the ability to handle file I/O and derived
classes.
It is only required to implement the two abstract methods
\vspace*{0.5cm}
\begin{code}
  abstract public Object parse( Field field, Class<?> cl, String valueString);

  abstract public String formatAsString( Object obj);
\end{code}

\vspace*{-0.25cm}
of \icode{ALDStandardizedDataIOCmdline}.

The first method should instantiate an object of class \icode{'cl'} from
the \icode{'valueString'} quite similar to the method \icode{readData(\ldots)}
introduced above, however, making no prior interpretation regarding a file to
use or derived class to return.
This has already been handled by the class \icode{ALDStandardizedDataIOCmdline}
prior to calling \icode{parse(\ldots)}.
Likewise, the method \icode{formatAsString(\ldots)} should always return a textual
representation of the given object as return value.
If it is necessary to use formatting information provided as part of
the argument \icode{'locationString'} of \icode{writeData(\ldots)}, the method
\vspace*{0.5cm}
\begin{code}
  public String formatAsString( Object obj, String formatString)
\end{code}

\vspace*{-0.25cm}
may be overridden.

As mentioned, I/O managers may give providers hints to adapt their
functionality.
In the case of the command line context this is the request to
read or write a history file in case the I/O is to be performed from or
to file.
The method \icode{isDoHistory()} may be used to inspect the state of the manager.
The standardized command line provider adheres to this standard.
