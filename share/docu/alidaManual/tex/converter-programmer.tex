\subsection{Automatic data type conversion}
\label{subsec:converter}

\alida features a generic mechanism for data type conversion.
This conversion mechanism is currently used within the graphical workflow editor \grappa
(see Sec.~\ref{subsec:grappa}).
\grappa allows to connect an output port of one operator to the input port of another operator
if the data types of the underlying parameters are compatible.
The data types are defined as compatible if either the target data type is
assignable from the source data type according to the Java class hierarchy or
an appropriate data type converter is supplied.
For example \alida ships with the converter 
\icode{ALDNumberConverter} between numeric data types, obviously
with potential loss of precision.
Likewise conversion from vectors to 1D arrays is supported
by the class \icode{ALDVectorNativeArrayConverter}.

Similar to data I/O providers, data type converters are registered on start-up by 
a singleton instance of the class \icode{ALDDataConverterManager}.
This manager redirects conversion requests to appropriate
converters.
This requires each data type converter to implement the interface
%%\vspace*{0.5cm}
\begin{code}
de.unihalle.informatik.Alida.dataconverter.ALDDataConverter
\end{code}
and to be annotated with the \alida annotation
\icode{@ALDDataConverterProvider}.
In analogy to \alida's data I/O provider a priority may be specified which is used
by the converter manager to resolve competion between multiple converters for a
given pair of data types to be converted.

A data type converter has to implement the methods
%%\vspace*{0.5cm}
\begin{code}
Collection<ALDDataConverterManager.ALDSourceTargetClassPair> providedClasses()
boolean supportConversion(Class<?> sourceClass, Type[] sourceTypes, 
                               Class<?> targetClass, Type[] targetTypes)
\end{code}
The first one is called upon start-up and it has to return pairs of source and
target data types which the converter is able to handle.
However for parameterized types this only indicates that the converter can in principle handle conversion 
for these classes but depending on the type parameters still may refuse to do so.
The second method, \icode{supportConversion} states precisely if conversion is supported
taking also type parameters passed as arguments into account.

The actual conversion is performed invoking the method
\begin{code}
Object convert(Object sourceObject, Type[] sourceTypes,
               Class<?> targetClass, Type[] targetTypes)
\end{code}
As stated above this method is typically not invoked directly but via the 
\icode{ALDDataConverterManager}.

Note that both converters supplied by \alida are implemented as an \icode{ALDOpertor}.
Thus the conversions are reflected within the processing history unless
they are intentionally hidden (see Sec.~\ref{subsec:operators-programming} and \ref{subsec:history}).
