\subsection{Commandline user interface}
\label{subsec:userCmdline}


The command line user interface of \alida allows to invoke all \alida operators
properly annotated to grant generic execution.
In the following examples the operators \icode{MatrixSum}
and \icode{ApplyToMatrix},
included as demo operators in \alida, are used to explain the usage and
features of this user interface.

\paragraph{Basics.}

You may invoke an operator by calling the command line operator runner:
\vspace*{0.5cm}
\begin{code}
java de.unihalle.informatik.Alida.tools.ALDOpRunner [-u] [-v] [-n] [-r] [-s] \
	  <classname> {parametername=valuestring}*
\end{code}

\vspace*{-0.25cm}
It expects as arguments the name of the operator class to be executed and its
parameters. The following options are available:
\begin{itemize}
  \item \icode{-u} / \icode{--usage} \hspace*{0.5cm} --- \hspace*{0.5cm}
  	prints the help
  \item \icode{-v} / \icode{--verbose} \hspace*{0.5cm} --- \hspace*{0.5cm}
  	prints additional information during operator execution
  \item \icode{-n} / \icode{--donotrun} \hspace*{0.5cm} --- \hspace*{0.5cm}
    does not execute the operator, but rather prints its short \\
    \hspace*{4.8cm}documentation (if available) and the parameters
  \item \icode{-r} / \icode{--useRegEx} \hspace*{0.5cm} --- \hspace*{0.5cm}
        interprets the \icode{classname}  as a regular expression
  \item \icode{-s} / \icode{--showProgress} \hspace*{0.5cm} --- \hspace*{0.5cm}
        enables the display of progress messages on console \\
        \hspace*{5.65cm}during operator execution
  \item \icode{--noDefaultHistory} \hspace*{0.5cm} --- \hspace*{0.5cm}
        if this flag is given, no history is read or written for input\\
        \hspace*{4cm}and output parameters contrarily to \alida's standard
        behaviour.
\end{itemize}
The operator \icode{MatrixSum} for row-wise summation of the elements within a
matrix can be called as follows:
\vspace*{0.5cm}
\begin{code}
java de.unihalle.informatik.Alida.tools.ALDOpRunner MatrixSum \
      matrix='[[1,2,3],[4,5,6]]' sums=-
\end{code}

which returns as result on standard output
\begin{code}
sums = [6.0,15.0]
\end{code}

Parameter values are specified as name=value pairs.
\alida's syntax for 2D array should be self-explanatory  from this example.
As the mode of summation is not supplied as a parameter its default is used.

The name of operators need not to be fully qualified if  they remain
unambiguous.
There are further options for abbreviation as well as regular expressions, see the help message of
\icode{ALDOpRunner}.

Note, that the command
\begin{code}
java de.unihalle.informatik.Alida.tools.ALDOpRunner MatrixSum matrix='[[1,2,3],[4,5,6]]' 
\end{code}
will return no output at all as the command line user interface returns only  those output parameters which have been requested by the user.
This is facilitated providing a dummy value for an output parameter, which is
\icode{-} in the example.

The enumeration defined in \icode{MatrixSum} 
for \icode{summarizeMode} 
is set in the next example.
If a wrong value for an enumeration is given the \icode{ALDOpRunner}
prints a list of admissible values.
The example also demonstrates redirection of output
to a file, \icode{sums.out} in this case, which is the standard in \alida if the
value of an output parameter is preceded with a '\icode{@}':\\[0.25cm]
\begin{code}
java de.unihalle.informatik.Alida.tools.ALDOpRunner MatrixSum matrix='[[1,2,3],[4,5,6]]' 
	summarizeMode=COLUMN sums=@sums.out
\end{code}

Input can be read from file as well:
\begin{code}
java de.unihalle.informatik.Alida.tools.ALDOpRunner MatrixSum matrix=@data sums=-
\end{code}

where the file data contains the string defining the matrix, e.g.,
\icode{[[1,2,3],[4,5,6]]}


\paragraph{Operators as parameters}
The demo operator \icode{ApplyToMatrix} takes as one parameter another
operator.
In this case this operator needs to extend the abstract class \icode{ALDSummarizeArrayOp}.
When invoking the \icode{ApplyToMatrix} operator from command line
we thus have to handle derived classes as value for parameters.
In the graphical user interface \alida features a combo box where
we may choose from.
In the command line interface \alida allows to prefix the value of a parameter
with a derived class to be passed to the operator.
This is necessary as \alida has, of course, no way to itself
decide if and which derived class is to be used.
\alida's syntax is to enclose the class name in a dollar sign and a colon.
This mechanism is not confined to operators as a parameter (of another operator),
but applies to all classes for which data I/O is handle in a standardized way
(see Sec.~\ref{subsubsec:implProviderCmdline}).
The value of the parameter behind the specified dervied class
is empty in this example as the operator \icode{ALDArrayMean}
can be instantiated without arguments.
As evident in the following example, abbreviations of the fully
qualified class name are accepted as long as they are unambiguous.
\begin{code}
java de.unihalle.informatik.Alida.tools.ALDOpRunner ApplyToMatrix \
	matrix='[[1,2,3],[4,5,6]]' \
	summarizeMode=ROW \
	summarizeOp='$ALDArrayMean:' \
	summaries=-
\end{code}
results in
\begin{code}
summaries = [2.0,5.0]
\end{code}

In case the operator, in this case \icode{ALDArrayMean}, is to be instantiate with additional arguments, this may be accomplished in curly brackets behind the specification of the dervided class:
\begin{code}
java de.unihalle.informatik.Alida.tools.ALDOpRunner ApplyToMatrix \
	matrix='[[1,2,3],[4,5,6]]' \
	summarizeMode=ROW \
	summarizeOp='$ALDArrayMean:{verbose=true}' \
	summaries=-
\end{code}
The result is the same in this case, as the standard verbose flag of all operators
has no effect in this case. The syntax to pass arguments for instantiation to 
an operator as a parameter is identical to the syntax of parameterized classes
and is described in the next paragraph.
If arguments are to be passed to an operator without using
a derived class this may be accomplished using only curly brackets, again identical
to parameterized classes. 
In this example this would yield
\begin{code}
java de.unihalle.informatik.Alida.tools.ALDOpRunner ApplyToMatrix \
	matrix='[[1,2,3],[4,5,6]]' \
	summarizeMode=ROW \
	summarizeOp='{verbose=true}' \
	summaries=-
\end{code}
which however fails, as the \icode{ALDSummarizeArrayOp} is abstract.

Note, that a dervided class is only available in the command line interface
if it is programmatically annoted with \icode{@ALDDerivedClass}, see Sec.~\ref{subsubsec:implOperators-advanced}.


\icode{ALDOpRunner} may be persuaded to show all operators derived from \icode{ALDSummarizeArrayOp}
and known within the user interface if we enter an invalid class name:
\begin{code}
java de.unihalle.informatik.Alida.tools.ALDOpRunner ApplyToMatrix \
        matrix='[[1,2,3],[4,5,6]]' \
	summarizeMode=ROW summarizeOp='$dd:' \
	summaries=-
\end{code}
yields
\begin{code}
ALDStandardizedDataIOCmdline::readData found 0 derived classes matching <dd>
      derived classes available:
	de.unihalle.informatik.Alida.demo.ALDArrayMean
	de.unihalle.informatik.Alida.demo.ALDArrayMin
	de.unihalle.informatik.Alida.demo.ALDArraySum
\end{code}


Supplemental parameters are handled like other parameters
\begin{code}
java de.unihalle.informatik.Alida.tools.ALDOpRunner ApplyToMatrix \
	matrix='[[1,2,3],[4,5,6]]' \
	summarizeMode=COLUMN \
	summarizeOp='$ALDArrayMin:{}' \
	summaries=- \
	returnElapsedTime=true \
	elapsedTime=-
\end{code}
gives
\begin{code}
	summaries = [1.0,2.0,3.0]
	elapsedTime = 4
\end{code}


\paragraph{Parameterized classes}
\alida supports so called \textit{parameterized classes}.
A parameterized class is essentially just an ordinary class where
however some member fields have been declared 
to be required
for an object of this class to be properly instantiated by \alida.
These member fields resemble quite some analogy to parameters of operators
and share some properties (see Sec.~\ref{subsubsec:implOperators-advanced} 
for details).
The syntax for parameterized classes is a comma separated list of name=value pairs
enclosed in curly brackets where names refer to annotated member variables of
the parameterized class.
This is shown for the class \icode{ExperimentalData1D}
which holds an 1D array of experimental data and descriptive text as annotated member fields.

\begin{codecsh}
java de.unihalle.informatik.Alida.tools.ALDOpRunner SmoothData1D \
	experiment='{ baselineCorrected=false , \
		description="my experiment", \
			data=[1.0,2.0,2.2,3.3,2.0,1.0,1.0,1.0,1.0,2.0,3.3,2.0]}' \
	smoothingMethod=MEAN width=3 \
	smoothedExperiment=-
\end{codecsh}
yields
\begin{code}
smoothedExperiment = { baselineCorrected=false , 
    data=[1.5,1.73,2.5,2.5,2.1,1.33,1.0,1.0,1.33,2.10,2.43,2.65] , 
    timeResolution=NaN , description="my experiment" (smoothed) }
\end{code}

If a class derived from \icode{ExperimentalData1D} was to be supplied to the operator,
the curly brackets can be prefixed by a derive class definition starting with a dollar sign
and ending with a colon as shown for the summarizing operators above.

\newpage
\paragraph{Advanced examples} ~

\TODO{some more elaborate examples}

The following example shows, that the standard sytnax used  for file I/O
may be nested
\begin{code}
java de.unihalle.informatik.Alida.tools.ALDOpRunner SmoothData1D \
        experiment='{data=@myexp.data,description="Demo experiment"}' \
        smoothedExperiment=@Exp1-smooth.txt
\end{code}
Here, the parameter \icode{description} of the parametrized class \icode{ExperimentalData} 
is directly parsed from the string given on the command line, while
the parameter \icode{data} is parsed from the content of the file \icode{myexp.data}.
The resulting \icode{smoothedExperiment=@Exp1} is written to the file \icode{Exp1-smooth.txt}.

Likewise, if an output parameter is a parametrized class, as in this example, a subset of
its parameters may be written to file, a part to standard output:
\begin{code}
java de.unihalle.informatik.Alida.tools.ALDOpRunner SmoothData1D \
        experiment='{data=@myexp.data,description="Demo experiment"}' \
        smoothedExperiment='{data=@myexp-smooth.data,description=-}'
\end{code}
will return
\begin{code}
smoothedExperiment = { data written using @myexp-smooth.data , 
          description="Demo experiment" (smoothed) }
\end{code}
which indicates that the data of the smoothed experiment have been written
to the file \icode{@myexp-smooth.data} as requested.

