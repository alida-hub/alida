\subsection{\alida operators}
\label{subsec:operators-programming}

\subsubsection{Using operators}
\label{subsec:using}
To use an operator an object of the operator class needs to be instantiated,
and input data as well as parameters have to be set for this object.
Subsequently, the operator can be invoked using the method
\icode{runOp()}.
After return from that routine the results can be retrieved from the operator.

\textbf{Important note:} Do no invoke an operator directly by its \icode{operate()} method
as this will prevent the processing history from being constructed.
Anyway, this only would be feasible from within the package of the operator
as the abstract method \icode{operate()} is declared \icode{protected}.

An example of how to use an operator is given in Fig.~\ref{exa:useop}.
First, a new instance of the operator is created (line 1), and subsequently
input parameters are set (lines 2 and 3).
If all required input parameters have been assigned for the operator object,
it can be invoked calling its \icode{runOp()} method (line 4). 
Upon invocation of \icode{runOp()}  the validity of input parameters is checked. 
Validity requires for an operator that all required input parameters
have values different from '\icode{null}'.
In addition, the implementation of an operator may impose further constraints
which, e.g., may restrict the admissible interval of numerical parameters
(see Sec~\ref{subsubsec:implOperators-advanced}).
Subsequent to successful validation, the method 
\icode{operate()}
is invoked. Each operator is supposed to implement this method as it does the
actual work. After return from \icode{runOp()}, the
resulting output data can be retrieved from the operator either directly accessing the member
variables or by getter methods as provided by the specific operator.
Note that the value of the operator parameters may
have changed upon return from \icode{runOp()} due 
to modifications in the \icode{operate()} method.
\icode{runOp()} may throw an exception if validation of inputs and parameters or
data processing itself fails.\\

\begin{figure}[hb]
\lstinputlisting[xrightmargin=.\textwidth,
xleftmargin=.05\textwidth,frame=single]{\demoCodeDir/NormalizeExperimentalDataOp-use.snippet}
\caption{\label{exa:useop}An example of how to use an operator, in this case
	\icode{NormalizeExperimentalDataOp}.}
\end{figure}

For the \icode{runOp()} method two version are available.
Besides the one mentioned above without arguments,
the method \icode{runOp(hidingMode)} is available where the value of
\icode{'hidingMode'} influences the visibility of the operator invocation in the
history.
If \icode{'hidingMode'} is \icode{VISIBLE} then the invocation of the operator
is visible.
If the value is \icode{HIDDEN} the invocation of the operator and all is children is hidden.
Finally, if \icode{hidingMode} is set to \icode{HIDE\_CHILDREN} the operator itself is visible,
but all its children are hidden from the history
See Section~\ref{subsec:history} for more details.
\TODO{more explanatiion?, example(s)}

An operator object may be reused to invoke processing
several times as long as input parameters are changed between
subsequent calls of \icode{runOp()}.

\subsubsection{Implementing operators: Basics}
\label{subsubsec:implOperators-basics}

Each operator in \alida is implemented by extending the abstract class \icode{ALDOperator}.
The example given in Fig.~\ref{exa:defineOp}
is the implementation of the demo operator \icode{MatrixSum},
included in the package \icode{de.unihalle.informatik.Alida.demo}.
It
shows
that an operator usually has to be annotated with the \texttt{@ALDAOperator}
annotation provided by \alida.
Some functionality of \alida, most importantly the  execution via automatically generated user interfaces,
 requires this annotation
to register the class as an \alida operator.
If an operator is annotated with \texttt{@ALDAOperator}, a public standard
constructor has to be supplied (see below), otherwise a compilation error will result.
Note that abstract classes can not be annotated with \texttt{@ALDAOperator}.

An operator may declare its preferences for generic execution, i.e., whether to
be or not to be generically executed, by using the parameter
\icode{'genericExecutionMode'} of the annotation. It currently supports four
possible values:
\begin{itemize}\itemseparation{0.1em}
\item \icode{NONE (default)}, to prohibit generic execution completely
\item \icode{SWING}, to allow generic execution via GUI only,
\item \icode{CMDLINE}, to allow generic execution via command line only, and
\item \icode{ALL}, to allow for generic execution in general.
\end{itemize}

Furthermore, operators can be categorized into \icode{'STANDARD'}
or \icode{'APPLICATION'} using the parameter \icode{'level'}.
The latter one is intended to subsume only operators that can easily be applied by non-expert users,
while the first category subsumes all operators. The graphical operator runner included in \alida 
provides two different view modes for either only operators annotated as
\icode{'APPLICATION'}, or all operators registered according to the
\texttt{@ALDAOperator} annotation.

Finally, the annotation also allows to release an operator for batch
processing. If the parameter \icode{'allowBatchMode'} is set to \icode{'true'},
which is the default, the operator control frame triggered by the GUI operator
runner will show the batch mode tab, and it is expected that the operator
behaves reasonable in that mode. If the parameter is set to
\icode{'false'}, batch mode execution will not be possible at all.

\begin{figure}[h]
\lstinputlisting[xrightmargin=.0\textwidth,
xleftmargin=.05\textwidth,frame=single]{\demoCodeDir/MatrixSum-declare.snippet}
\caption{\label{exa:defineOp}Example deriving the operator \icode{MatrixSum}.}
\end{figure}

There are five issues which have to be taken care of when implementing
an operator, namely
\begin{itemize}\itemseparation{0.1em}
\item	to define the parameters of the operator,
\item	to implement the functionality of operation per se,
\item 	to provide constructors, particularly a public one without arguments
\item	to optionally constrain admissible parameter values,
\item 	and to indicate whether
          this operator prefers a complete processing history
	or a processing history according to data dependencies.
\end{itemize}
The first three issues
are described in the following,
while the last two are deferred to Sec.~\ref{subsubsec:implOperators-advanced}.

\paragraph{Parameters.}
The common way to define the parameters of an operator is by annotation of
corresponding member variables. In addition, since \alida 2.5 it is also possible to
dynamically add and remove parameters via methods provided by \icode{ALDOperator}. For defining
parameters via annotations currently a modified version of the annotation \icode{@Parameter} as under development for ImageJ
2.0$\,$\footnote{ImageJ 2.0 project, \url{http://developer.imagej.net/about}} is used.
The relevant fields of this annotation are listed below and will be detailed in
the following:
\begin{itemize}\itemseparation{0.1em}
\item	\icode{direction} --- direction of the parameter, i.e., \icode{IN},
\icode{INOUT} or \icode{OUT}
\item	\icode{required} --- flag to mark required parameters
\item	\icode{label} --- custom name of parameter
\item	\icode{description} --- short descriptive explanation
\item	\icode{supplemental} --- flag to mark supplemental parameters
\item	\icode{mode} --- importance category of the parameter
\item	\icode{dataIOOrder} --- I/O rank among all parameters of the operator
\item	\icode{callback} --- name of a callback method to be automatically
 invoked upon changes of the parameter's value
%\item	\icode{info} --- parameter is a dummy parameter and exclusively used for
% layout of user interfaces
\item	\icode{modifiesParameterDefinitions} --- changing the parameter's value
 may add or remove parameters of this operator
\end{itemize}

Fig.~\ref{exa:defineParameters} shows an example how parameters are defined this way.
If the \icode{'direction'} of a parameter is set to \icode{'IN'} or
\icode{'INOUT'}, the field \icode{'required'} defines whether this parameter is
required or optional. 
The field \icode{'description'}
of the parameter gives a textual explanation, and the \icode{'label'} may be
used for display purposes.
Setting the field \icode{'supplemental'} to \icode{'true'} declares the
corresponding parameter as supplemental.
Via the Java inheritance mechanism an operator inherits all parameters defined
in its super classes.

\begin{figure}
\lstinputlisting[xrightmargin=.\textwidth,
xleftmargin=.05\textwidth,frame=single]{\demoCodeDir/MatrixSum-parameters.snippet}
\caption{\label{exa:defineParameters}Example defining the parameters of \icode{MatrixSum}.}
\end{figure}

The annotation parameter \icode{'dataIOOrder'} allows to rank parameters
in interface generation. For example, in GUI generation it might be favorable to place the most important 
parameters on top of the window, while parameters of minor importance only
appear at the bottom. Likewise, in command line tools some parameters might be
supposed to appear earlier in the help system than others.
Such a ranking can be achieved by specifying an I/O order. Smaller values refer
to a high importance of the corresponding parameter, larger values to minor importance.
%If a parameter is annotated with \icode{'info'} as \icode{true} the parameter
% is a dummy parameter with no value associated. Rather it is exclusively used to layout automatically
%generated user interfaces. Currently only \icode{'info'} parameters of class
% \icode{String} are supported.

\alida allows to categorize operator parameters according to the level
of knowledge required for their use. Often some parameters of operators are only of interest for experts, and non-expert users
do not even have to be aware of them. To this end each parameter may be
annotated as \icode{'STANDARD'} or \icode{'ADVANCED'}. Accordingly, \alida's graphical user
interface allows to switch the view of parameters between showing only standard parameters and
displaying all.

The annotation parameter \icode{'callback'} allows to specifiy the name of a method of this operator
which is to be automatically invoked by \alida if the annotated parameter's value is changed.
Note the callback method is invoked automatically only when changing the
parameter value using the \icode{setParameter()} method. Callback functions offer the possibility to
dynamically reconfigure operators depending on the current context. In particular it is possible to
add and remove parameters dynamically. Consider for example an operator allowing for different types
of inputs, i.e.~arrays and lists. Each of these types requires specific treatment in data
I/O, i.e.~the graphical user interface needs to provide two different graphical elements. Thus,
to handle the different parameters one option would be to add two parameters of the two different
types to the operator and always display both (although only one is needed at a time). Using
callbacks a more elegant way to solve the problem is available. We can add another parameter
\icode{'input mode'} to the operator which allows to select the type of input. If we furthermore
define a callback function for that parameter the user interface of the operator can dynamically be
reconfigured. Depending on the chosen input mode the corresponding input parameter can be added
and the second one be removed dynamically. Adding and removing parameters can be accomplished with
the methods \icode{'addParameter()'} and \icode{'removeParameter()'} of \icode{ALDOperator}.

Note that changing the set of parameters of an operator
dynamically requires \alida to update its internal representation and also graphical user interfaces
attached to the operator. To this end the programmer is requested to let \alida know that a callback
function changes the parameter set of an operator. This is to be accomplished using the
annotation parameter \icode{modifiesParameterDefinitions} which is to be set to \icode{'true'} or
\icode{'false'}. Note that if the parameter is set to false and a callback function nonetheless 
modifies the parameters of an operator, undefined behaviour of the operator may result.

\paragraph{Operator functionality.}
The method \icode{operate()} implements the functionality of the operator. All data
passed into and returned from the operator have to be passed through the parameters of the operator.
They may be set and retrieved with the generic
\icode{getParameter(name)} and \icode{setParameter(name,value)} methods
of \icode{ALDOperator}, by specific setter- and getter-methods as provided by
each operator, or by
directly accessing the member fields if allowed.
To invoke the processing within an operator, i.e., to run its \icode{operate()}
routine, the final method \icode{runOp()} supplied by \icode{ALDOperator} needs to be called by the user of an operator.

\paragraph{Constructors.}
As noted above, 
for automatic code generation and documentation capabilities as well as generic execution 
of an operator,
the operator class needs to implement 
a public standard constructor 
as shown in Fig.~\ref{exa:constructor}.
This is, however, not necessary if the operator is only to be used explicitly
by the programmer and is not annotated.
Further convenience constructors may be implemented which additionally set
parameters.

\begin{figure}
\lstinputlisting[xrightmargin=.\textwidth,
xleftmargin=.05\textwidth,frame=single]{\demoCodeDir/MatrixSum-constructor.snippet}
\caption{\label{exa:constructor}Constructors of \icode{MatrixSum}.}
\end{figure}

\paragraph{Processing history.}
As decribed in Sec~\ref{subsec:using} the visibility of an operator invocation in the
processing history may be
influcenced by the parameters of the  \icode{runOp()} method.
By this means the visibility of each operator invocation may be set to
icode{VISIBLE}, \icode{hidingMode}, or \icode{HIDE\_CHILDREN}.
This only allows to determine the visibility of further operators directly
invoked by their \icode{runOp()} method but not of operators indirectly invoked.
Thus, the implementor of an operator may influence the visibility
of its own invocation using the method \icode{setHidingMode(hidingMode)}.
The main usage of this method is to set the operators visibility to \icode{HIDE\_CHILDREN}
to hide recursive calls to further operators from the history.
This also hides operator calls which are indirectly invoked arbitrary methods used when implementing
the \icode{operate()} method.

\textbf{Important note:} It is strongly recommended that an operator
does not rely on specific initializations, e.g., of private fields, that are
performed in a constructor and depend on the values of input parameters. Rather,
it is advised that upon invocation of the \texttt{operate()} method an operator performs all necessary initializations according to the parameter settings.
This allows to take into account changes of parameter values subsequent to
construction of the operator object by using the \texttt{setParameter(\ldots)}
method or dedicated setter methods.
Otherwise generic execution of the operator is not feasible and
the operator should not be released for generic execution.

\begin{figure}[tb]
\lstinputlisting[xrightmargin=.\textwidth,
xleftmargin=.05\textwidth,frame=single]{\demoCodeDir/ApplyToMatrix-parameters.snippet}
\caption{\label{exa:operatorAsParameter}An example of an operator as a parameter in the operator \icode{ApplyToMatrix}}
\end{figure}

\subsubsection{Implementing operators: Advanced techniques}
\label{subsubsec:implOperators-advanced}

\paragraph{Operators as parameters.}
\alida supports as parameters of an operator also any (other) \alida operator.
This is shown for the demo operator \icode{ApplyToMatrix} in Fig.~\ref{exa:operatorAsParameter}.
In this case, the class \icode{ALDSummarizeArrayOp} is an abstract operator 
which takes a 1D array as input and returns a summarizing scalar.
Now, one may implement concrete examples of such a summarizing operation. 
As examples \alida ships with operators implementing
the summation (\icode{ALDArraySum}), mean (\icode{ALDArrayMean}) and minimum
(\icode{ALDArrayMin}) operations.
Each operator implements the  \icode{operate()} method and has to supply a
standard constructor.
As shown in Fig.~\ref{exa:ALDArraySum},
the  operator \icode{ALDArraySum} is declared as operator on the
\icode{'STANDARD'} in contrast to \icode{'APPLICATION'} level, as
it is not expected to be invoked as an application.
However, setting the level to standard in the menu of the graphical user interface
stills allows their direct execution.
When extending the super class \icode{ALDSummarizeArrayOp}, it is necessary to
annotate the class with \icode{@ALDDerivedClass} in order to allow \alida's data
I/O mechanisms to find the derived class in automatically generated user
interfaces.
This holds for other parameter types as well, see Sec.~\ref{subsubsec:datatypes}.

\begin{figure}
\vspace{4mm}
\lstinputlisting[xrightmargin=.\textwidth,
xleftmargin=.05\textwidth,frame=single]{\demoCodeDir/ALDArraySum-constructor.snippet}
\caption{\label{exa:ALDArraySum}The operator \icode{ALDArraySum} as an operator in standard level.}
\end{figure}

\paragraph{Constraints on admissible parameter values.}

The implementation of an operator may impose 
custom constraints on the input parameters
beyond the general requirement, that all required input parameters
need to have non-null values.
This is achieved by overriding the method \\
\begin{lstlisting}[xrightmargin=.00\textwidth, xleftmargin=.0\textwidth,frame=single,numbers=none]
  public void validateCustom() throws ALDOperatorException
\end{lstlisting}
which, e.g., may restrict the admissible interval of numerical parameters.
This method is called by \icode{runOp()} prior to the invocation of the
\icode{operate()} method of an operator. It is supposed to throw an exception
of type \icode{ALDOperatorException} with type \icode{VALIDATION\_FAILED} and
a striking error message if the validation was not successful.

\paragraph{Preference for history graph construction.}
Each operator may specify a preferred
way to create the processing history by setting
its member variable \icode{'completeDAG'}.
The default mode is a complete processing history,
i.e., \texttt{completeDAG == true}. This works in all cases,
but potentially includes additional operator invocations as performed in the
\icode{operate()} method which do not directly influence
the values of the object for which the history is constructed.
To generate a leaner history graph the programmer of an operator
may choose to set \icode{'completeDAG'} to \icode{false}, see
Section~\ref{subsec:history} for details.

In case an implementor decides to completely disable the construction of the processing history 
this can be accomplished with the static method \icode{setConstructionMode()} of
the \icode{ALDOperator} class.


\paragraph{Progress events.} \alida provides a mechanism for operators
to send status and progress messages to the user during execution. These messages
are shown in the status bar of the operator control windows
(Sec.~\ref{subsubsec:controlWin}) and also in Grappa's log panel
(Sec.~\ref{subsec:grappa}). Accordingly, in constrast to output to standard out,
which is not guaranteed to reach the user, these messages will always be visible
to the user. For triggering such messages, an operator has to fire an event of
type \icode{ALDOperatorExecutionProgressEvent} using the method\\
\begin{lstlisting}[xrightmargin=.00\textwidth, xleftmargin=.0\textwidth,frame=single,numbers=none]
  protected void
	  fireOperatorExecutionProgressEvent(ALDOperatorExecutionProgressEvent ev)
\end{lstlisting}
The event takes a message string as argument during construction which is then
displayed to the user.

\subsubsection{Implementing operators: Parameters}
\label{subsubsec:datatypes}

\paragraph{Derived classes.}
As noted above, 
if an instance of a class is to be supplied in an automatically
generated user interface as a value for a parameter of one of its super classes,
\alida requires the annotation \icode{@ALDDerivedClass}.

\paragraph{Parameterized classes.}
\alida provides automatic I/O of primitive data types, enumerations, arrays, collections,
and operators.
In addition, so called \textit{parameterized classes} are supported.
Any Java class may be declared to be a parameterized class in \alida
by annotating the class with \icode{@ALDParametrizedClass} as shown for the
class \icode{ExperimentalData} in Fig.\ref{exa:parametrizedClass}.
The class is annotated by \icode{@ALDParametrizedClass},
and all members to be configured via \alida's user interfaces are
to be annotated with \icode{@ALDClassParameter}. This annotation offers the
following arguments:
\begin{itemize}
  \item \icode{'label'} --- name of the parameter, field has same
	semantics as for parameters of operators
  \item \icode{'dataIOOrder'} --- I/O rank of the parameter	 
  \item \icode{'mode'} ---
	importance of the parameter, i.e., \icode{'STANDARD'} or \icode{'ADVANCED'}	 
  \item \icode{'changeValueHook'} --- post-processing routine after reload (see
  below)
\end{itemize}

The first three annotations share their semantics with their counterparts in the
\icode{@Parameter} annotation of operators
(cf.~Sec.\ref{subsubsec:implOperators-basics}).
Only the forth one is special to parameterized classes. On loading or saving
operator configurations, i.e.~the values of its parameters, also objects of
parameterized classes used as parameters have to be handled. To this end these
are also converted to XML format. However, in doing so only {\em annotated}
member variables of the parametrized class are taken into account. Consequently,
also only these member variables of the object will get proper values after
reload.
In some cases, however, this might yield to inconsistencies in an object's state
after reload. Consider for example a parameterized class representing a set of
points. Most likely this class will define a member variable of a list type
which holds the points, and as this member holds the core data of the data
type, it will be annotated.
Now, for efficiency of implementation it might be of advantage to define another
member variable keeping track of the number of points in the list, usually not
being annotated. To ensure that this variable is also correctly set on reload of
the object, the \icode{'changeValueHook'} argument can be used. It expects a
string defining a method of the class at hand which is to be called after all
annotated member variables have been properly set during reload. The method may
then perform some kind of post-processing of the object's configuration, i.e.,
set the point counter variable to the number of points found in the given list.
Note that the hook method should not expect any arguments.

The above annotations are
the only implementational overhead to allow \alida to automatically generate user interfaces where the parameterized class acts as a parameter.
Operators acting as parameters of other operators may be viewed as
special cases of parameterized classes.


\begin{figure}
\lstinputlisting[xrightmargin=.\textwidth,
xleftmargin=.05\textwidth,frame=single]{\demoCodeDir/ExperimentalData.snippet}
\caption{\label{exa:parametrizedClass}The class \icode{ExperimentalData} as an example of a parameterized class.}
\end{figure}

\paragraph{Automatic documentation}

The value of each \icode{'IN'} and \icode{'INOUT'} parameter is recorded
upon invocation of an operator via its \icode{runOp()} method using the method \icode{toString()}
of the parameter class for later storage in the processing history.
Thus, it is recommended to supply an appropriate \icode{toString()}
method for data types used as parameters to yield informative histories.

If a parameter of an operator is expected to be documented in the data flow of
the processing history, it
may be of any Java class being uniquely
identifiable. This excludes only primitive data types, interned strings
and cached numerical objects.
If the parameter need not to be part of the data flow, all classes are
acceptable.

Note that before returning from \icode{runOp()}, additional documentation is
done for output objects derived from the abstract class \texttt{ALDData}.
This class essentially features a property list which may 
be used to augment data objects, e.g., by a filename or URL specifying 
the origin of the data.
