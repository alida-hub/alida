\subsection{\alida operators}
\label{subsec:operators-user}

The heart of \alida's concept are operators that implement all data analysis capabilities. 
Operators are the only places where data are processed and manipulated.
Examples for data to be manipulated are, e.g., experimental measurements,
sets of DNA sequences, or, for image analysis, images and sets of regions
comprising a segmentation result. An operator receives zero or more input objects
comprising all input data the operator is expected to work on.
Operators with zero inputs are operators which for example create a data object
for given parameters or read data from file.
Further input to an operator are parameters, which configure or
modify the processing on the input data.
Examples are the selection of alternative processing procedures, e.g., if
experimental measurements should be summarized by their mean or their median,
a mask-size of a filter to be applied to an image, or the maximal number
of iterations for a gradient descent algorithm.
The distinction of an input acting as input data or as a parameter is not
clear in all cases.
As an abstract example consider an operator which is 
to compute the scalar product of two vectors.
In this case, both vectors are most likely considered as input data.
However, if the operator is
to normalize 
a data vector by a scalar normalizing constant,
this scaling factor may either be considered as an input or a parameter.
Therefore, \alida does not distinguish between input data and input parameters.
However, parameters of an operator may be optional, required or supplemental.

An operator produces zero or more output objects  as the result of processing.
%The types of these objects are the same as for input
%objects since in virtually all cases an output of one operator may act as the input to
%other operators.
An operator with zero output objects will, e.g., write data to disk.

All input and output data are denoted as {\em parameters} in \alida.
The role of a parameter is identified by its direction,
which may be input (\icode{IN}) or output (\icode{OUT}).
In cases, where an input object should just be passed through the operator or is
destructively modified, this parameter has the direction input and output
(\icode{INOUT}). An example is a vector which is modified in place.


%The behavior of operators is controlled or configured via parameters.
%All Java classes as well as primitive data types are accepted as a parameter type.
%Typical examples for parameters are the size of a kernel or structuring element, 
%a filter mask, constants (or parameters)
%which weight energy terms, step width of optimizers, or the maximal number
%of iterations for a gradient descent algorithm.
%%A threshold may either be considered as an input object or a parameter of the operator.}

In addition to parameters providing the input data and configuration of the
operator, and output parameters representing the results of processing, an
operator may use supplemental parameters.  
By definition, the setting of supplemental parameters must not influence the
data processing nor the results returned as output data.
Consequently, supplemental parameters are not documented in the processing
history.
Examples for such parameters include flags to control output or debugging
information, and intermediate results produced by an operator.

The relevant features defined for parameters in \alida are the following:
\begin{itemize}\itemseparation{0.1em}
\item	the direction of the parameter, which may be \icode{IN, OUT} or \icode{INOUT},
\item	a boolean indicating whether the parameter is supplemental, 
\item	a boolean indicating whether the parameter is required or optional
	(which is only interpreted for non-supplemental \icode{IN and INOUT}
	parameters) 
\item	a label, e.g., used in the graphical user interfaces,
\item	a textual explanation of the parameter, for example appearing in tooltips,
\item   a data I/O order by which parameters can be ranked for generic GUI or
command line interface generation, and
\item   an expert mode which, e.g., allows to hide parameters for advanced configuration from non-expert users.
\item   the name of a callback function which is called if the parameter's value is changed using
	the \icode{setParameter()} method of the operator; the function can, e.g., be used to
	automatically update other parameter values or to even modify the set of parameters of the
	operator (refer to Sec.~\ref{subsubsec:implOperators-basics} for details)
\end{itemize}

The application of operators may be nested as one operator may call one or more
other operators.
At the top of this hierarchy we typically have appropriate user interfaces.
Their parameter settings are facilitated via files, GUIs, command line, or
via the console.
